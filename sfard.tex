%%%%%%%%%%%%%%%%%%%%%%%%%%%%%%%%%%%%%%%%%%%%%%%%%%%%%%%%%%%%
%%  This Beamer template was created by Cameron Bracken.
%%  Anyone can freely use or modify it for any purpose
%%  without attribution.
%%
%%  Last Modified: January 9, 2009
%%
% 
\documentclass[xcolor=x11names,compress]{beamer}
 \setbeamertemplate{section in toc}[sections numbered]
 \setbeamertemplate{subsection in toc}[subsections numbered]
% \usetheme{Boadilla}
% \usecolortheme{albatross}
\graphicspath{{Figures/}} 
%% General document %%%%%%%%%%%%%%%%%%%%%%%%%%%%%%%%%%
\usepackage{graphicx}
\usepackage{amsmath,amsfonts,amssymb,amscd,amsthm,xspace}
\usepackage{tikz}
\usepackage{bbm}
\usepackage{color}
% \usetikzlibrary{decorations.fractals}
\usepackage{datenumber}
%%%%%%%%%%%%%%%%%%%%%%%%%%%%%%%%%%%%%%%%%%%%%%%%%%%%%%
\usepackage[utf8]{inputenc}
% \usepackage[latin1]{inputenc}
\usepackage[norsk]{babel}

%% Beamer Layout %%%%%%%%%%%%%%%%%%%%%%%%%%%%%%%%%%
\useoutertheme[subsection=true,shadow]{miniframes}
\useinnertheme{default}
\usefonttheme{serif}
\usepackage{palatino}

\setbeamerfont{title like}{shape=\scshape}
\setbeamerfont{frametitle}{shape=\scshape}

\setbeamercolor*{lower separation line head}{bg=DeepSkyBlue4} 
\setbeamercolor*{normal text}{fg=black,bg=white} 
\setbeamercolor*{alerted text}{fg=red} 
\setbeamercolor*{example text}{fg=black} 
\setbeamercolor*{structure}{fg=black} 
 
\setbeamercolor*{palette tertiary}{fg=black,bg=black!10} 
\setbeamercolor*{palette quaternary}{fg=black,bg=black!10} 

\renewcommand{\(}{\begin{columns}}
\renewcommand{\)}{\end{columns}}
\newcommand{\<}[1]{\begin{column}{#1}}
\renewcommand{\>}{\end{column}}
%%%%%%%%%%%%%%%%%%%%%%%%%%%%%%%%%%%%%%%%%%%%%%%%%%


\begin{document}
%%%%%%%%%%%%%%%%%%%%%%%%%%%%%%%%%%%%%%%%%%%%%%%%%%%%%%
%%%%%%%%%%%%%%%%%%%%%%%%%%%%%%%%%%%%%%%%%%%%%%%%%%%%%%

\begin{frame}
\title{Anna Sfard: Dual Nature of Mathematical Conceptions}
\subtitle{Educatinal Studies in Mathematics {\bf 22}:1-36, 1991}
\author{
          Theo Brønstad \\
	Hannu Lyyjynen	 
        }
        \date{\today} 
	\begin{figure}
          \centering
 %       \includegraphics[width=24mm]{UiBmerke_grayscaleV8.eps}
        \end{figure}
\titlepage
\end{frame}
%%%%%%%%%%%%%%%%%%%%%%%%%%%%%%%%%%%%%%%%%%%%%%%%%%%%%%
%%%%%%%%%%%%%%%%%%%%%%%%%%%%%%%%%%%%%%%%%%%%%%%%%%%%%%
 \begin{frame}
\frametitle{Dualitet: operasjonell og strukturell konseptualisering}
\tableofcontents
\end{frame}
%
\begin{frame}
\section{\scshape De to epistemologiske synevinklene}
      Konseptualiseringen av de matematiske begreppene kan skje på  
vesentlig  to måter:
      det {\bf operasjonelle} og det {\bf strukturelle.} 
      
      De er tilsynelatende inkompatible, men kan i dypere bemerkelse 
forstås å {\bf komplementere} hverandre.
\end{frame}
%
%
\begin{frame}
      Den {\bf operasjonelle} konseptualiseringen, som oftest kommer først, 
betrakter matematiske objekter 
(tall, variabel, funksjon, symmetri, geometriske, osv.) 
som resultater av 
{\bf prosesser}, {\bf algoritmer} eller {\bf handlinger}.     

Eksempel: $ \mathbb{N}$
\end{frame}
%
%
\begin{frame}
      Den {\bf strukturelle} konseptualiseringen i et matematisk univers
fokuserer mer på ({\it en representasjon av}) 
de abstrakte objektene (tall, variabel, funksjon, symmetri, geometriske, osv.)
som {\bf selvstendige enheter}.  

Eksempel: $ \mathbb{N}$ \pause

     Denne modellen er oftest regjerende i modern matematikk, 
men kan være svært vanskelig å oppnå.   
 
    {\it Syntesen} krever et kvantumhopp over den onkologiske (eksistentielle)
 kløften mellom de to forskjellige tankemåtene.
\end{frame}
%
%
\begin{frame}
\section{\scshape  Veien frem til nye konsepter}
      I utviklingen til et nytt matematisk konsept (både i forsknings- og læringssammenheng) kan tre hierarkiske faser identifieres.
    \begin{enumerate}
    \item {\it interiorization}
    \item {\it condensation}
    \item {\it reification}
    \end{enumerate}
\end{frame}
%
%
\begin{frame}
   
      andre.
\end{frame}
%

 \begin{frame}
\begin{table}[]
\centering
%{konseptualisering}
\label{my-label}
\begin{tabular}{lllll}
{\bf strukturell} & --- & dualitet & --- & {\bf operasjonell }\\
objekt    &  &  &  & prosess, algoritm, aksjon   \\
            &  &ikke dikotomi  &  &               \\
            
  statisk          &  &  &  &  dynamisk            \\
  holistisk & &  & & sekvensiell \\
abstrakt & & & & algoritmisk \\
relasjonell & & & & instrumentell \\
dialektisk & & & & aksjonell \\
integrert & & & & detaljert
\end{tabular}
\end{table}
\centering
{\scshape Konseptualisering av (matematiske) begrepp}
\end{frame}
%
%
\begin{frame}
   \frametitle{\scshape Utviklingen av tallsystemene }
    \subsection{Eksempel: utviklingen av tallsystemene}
      
\end{frame}
%
\begin{frame}
   \frametitle{\scshape  Fra trigonometrin til hyperboliske funksjoner }
    \subsection{Eksempel: fra trigonometrin til hyperboliske funksjoner}
      $\cdots$
\end{frame}
%
%
\begin{frame}
  \frametitle{\scshape   Kjerneregelen i differentieringen}
    \subsection{Eksempel: kjerneregelen i differentieringen}
       
\end{frame}
%

%
\begin{frame}
    \subsection{Eksempel: egenverdier og -vektorer}
      \frametitle{\scshape Egenverdier og egenvektorer}
      {\bf Definisjon}: Om ligningen $A w = \lambda w$ holder i $\mathbb{C}^n$, kalles (de komplekse)
       tallene $\lambda \in \mathbb{C}$
       for   matrisens $A_{\scriptscriptstyle n\times n}$ {\it egenverdier} og  vektorene $w_{\scriptscriptstyle n\times 1} \ne 0$  {\it egenvektorer}
       tilsvarende egenverdi $\lambda$.
\end{frame}
%
%
\begin{frame}
    \subsection{Diskusjon}
   Diskusjontemaer: aspektene av de
   \begin{enumerate}
   \item geometriske
   \item notasjonelle
   \item det språklige
   \end{enumerate}   
   ingrediensene til oppfatningen av {\it modus operandi.}
\end{frame}
%

\end{document}
